% Options for packages loaded elsewhere
\PassOptionsToPackage{unicode}{hyperref}
\PassOptionsToPackage{hyphens}{url}
%
\documentclass[
]{article}
\usepackage{amsmath,amssymb}
\usepackage{iftex}
\ifPDFTeX
  \usepackage[T1]{fontenc}
  \usepackage[utf8]{inputenc}
  \usepackage{textcomp} % provide euro and other symbols
\else % if luatex or xetex
  \usepackage{unicode-math} % this also loads fontspec
  \defaultfontfeatures{Scale=MatchLowercase}
  \defaultfontfeatures[\rmfamily]{Ligatures=TeX,Scale=1}
\fi
\usepackage{lmodern}
\ifPDFTeX\else
  % xetex/luatex font selection
\fi
% Use upquote if available, for straight quotes in verbatim environments
\IfFileExists{upquote.sty}{\usepackage{upquote}}{}
\IfFileExists{microtype.sty}{% use microtype if available
  \usepackage[]{microtype}
  \UseMicrotypeSet[protrusion]{basicmath} % disable protrusion for tt fonts
}{}
\makeatletter
\@ifundefined{KOMAClassName}{% if non-KOMA class
  \IfFileExists{parskip.sty}{%
    \usepackage{parskip}
  }{% else
    \setlength{\parindent}{0pt}
    \setlength{\parskip}{6pt plus 2pt minus 1pt}}
}{% if KOMA class
  \KOMAoptions{parskip=half}}
\makeatother
\usepackage{xcolor}
\usepackage[margin=1in]{geometry}
\usepackage{graphicx}
\makeatletter
\newsavebox\pandoc@box
\newcommand*\pandocbounded[1]{% scales image to fit in text height/width
  \sbox\pandoc@box{#1}%
  \Gscale@div\@tempa{\textheight}{\dimexpr\ht\pandoc@box+\dp\pandoc@box\relax}%
  \Gscale@div\@tempb{\linewidth}{\wd\pandoc@box}%
  \ifdim\@tempb\p@<\@tempa\p@\let\@tempa\@tempb\fi% select the smaller of both
  \ifdim\@tempa\p@<\p@\scalebox{\@tempa}{\usebox\pandoc@box}%
  \else\usebox{\pandoc@box}%
  \fi%
}
% Set default figure placement to htbp
\def\fps@figure{htbp}
\makeatother
\setlength{\emergencystretch}{3em} % prevent overfull lines
\providecommand{\tightlist}{%
  \setlength{\itemsep}{0pt}\setlength{\parskip}{0pt}}
\setcounter{secnumdepth}{-\maxdimen} % remove section numbering
\usepackage{bookmark}
\IfFileExists{xurl.sty}{\usepackage{xurl}}{} % add URL line breaks if available
\urlstyle{same}
\hypersetup{
  pdftitle={Tarea 1},
  pdfauthor={álgebra lineal},
  hidelinks,
  pdfcreator={LaTeX via pandoc}}

\title{Tarea 1}
\author{álgebra lineal}
\date{2024-09-14}

\begin{document}
\maketitle

\section{Vectores}\label{vectores}

\section{Rectas y planos}\label{rectas-y-planos}

\section{Sistemas de ecuaciones}\label{sistemas-de-ecuaciones}

\begin{enumerate}
\def\labelenumi{\arabic{enumi}.}
\tightlist
\item
  Considere los siguientes sistemas de ecuaciones lineales: \[
  \begin{array}{cc}
    2x_1 + 3x_2 &= 5 \\
    x_1  - 4x_2 &= -3 \\
    2x_1 - 8x_2 & = -6
  \end{array} 
  \quad \quad \quad
  \begin{array}{cc}
    2x_1 + 3x_2 &= 5 \\
    x_1  - 4x_2 &= -3 
  \end{array}
  \] Determine el conjunto solución de cada uno de los sistemas ¿Son
  sistemas equivalentes?
\end{enumerate}

\section{Matrices}\label{matrices}

\begin{enumerate}
\def\labelenumi{\arabic{enumi}.}
\item
  Calcula los productos \(ABC\) y \(BA\) de las matrices, con \[
  \begin{array}{lcr}
  A=\begin{pmatrix}
   991 & 992 & 993 \\
   994 & 995 & 996 \\
   997 & 998 & 999
    \end{pmatrix}, \quad &
  B = \begin{pmatrix}
   12 & -6 & -2 \\
   18 & -9 & -3 \\
   24 &-12 & -4
   \end{pmatrix}, \quad &
  C = \begin{pmatrix}
   1 & 1  \\
   1 & 2  \\
   3 & 0 
   \end{pmatrix}
  \end{array}
  \]
\item
  \begin{enumerate}
  \def\labelenumii{\alph{enumii})}
  \tightlist
  \item
    Una matriz se dice \emph{idempotente} si \(A^2 = A\). Probar que \[
    B= \begin{pmatrix}
     2 & -3 & -5 \\
    -1 &  4 & 5  \\
     1 & -3 &- 4
    \end{pmatrix}
    \] es idempotente
  \end{enumerate}
\end{enumerate}

\begin{enumerate}
\def\labelenumi{\alph{enumi})}
\setcounter{enumi}{1}
\tightlist
\item
  Demuestre qu si \(A\) es idempotente, entonces \(B=I_n-A\) también es
  idempotente,
\item
  Si \(A\) y \(B\) son como en b) entonces demuestre que
  \(AB= BA = \mathbf{0}\)
\end{enumerate}

\begin{enumerate}
\def\labelenumi{\arabic{enumi}.}
\tightlist
\item
  Se dice que una matriz \(n\times n\) es \emph{involutiva} si y sólo si
  \(A^2 = I_n\).
\end{enumerate}

\begin{enumerate}
\def\labelenumi{\alph{enumi})}
\item
  Verifica que
  \(A=\begin{pmatrix} 0 & 1 & -1 \\ 4 & -3 & 4 \\ 3 & -3 & 4 \end{pmatrix}\)
  es involutiva
\item
  Demuestre que si \(A\) es involutiva entonces \(\frac{1}{2}(I_n+A)\) y
  \(\frac{1}{2}(I_n-A)\) son idempotentes y su producto es igual a la
  matriz cero.
\end{enumerate}

1- Para las siguientes matrices, encontrar la fórmula para \(A^{n}\) con
\(n\in\mathbb{N}\),

{[}

\begin{multicols}{2}
 A= \begin{pmatrix}
    \cos x       & -\mbox{sen} x \\
    \mbox{sen} x & \cos x
    \end{pmatrix} &
    
A= \begin{pmatrix}
    a       & 1 \\
    0       & a
    \end{pmatrix} 
\end{multicols}

{]}

\end{document}
