% ^^A CTAN:macros/latex/contrib/supported/blindtext
%\iffalse
%<*package>
%\fi
\def\blindfileversion{V2.0}
\def\blindfiledate{2012/01/06}
% ^^A For index and changes, run:
% ^^A makeindex -s gglo.ist -o blindtext.gls blindtext.glo
% ^^A makeindex -s gind.ist -o blindtext.ind blindtext.idx
%
% \CheckSum{1786}
%%
%% \CharacterTable
%%  {Upper-case    \A\B\C\D\E\F\G\H\I\J\K\L\M\N\O\P\Q\R\S\T\U\V\W\X\Y\Z
%%   Lower-case    \a\b\c\d\e\f\g\h\i\j\k\l\m\n\o\p\q\r\s\t\u\v\w\x\y\z
%%   Digits        \0\1\2\3\4\5\6\7\8\9
%%   Exclamation   \!     Double quote  \"     Hash (number) \#
%%   Dollar        \$     Percent       \%     Ampersand     \&
%%   Acute accent  \'     Left paren    \(     Right paren   \)
%%   Asterisk      \*     Plus          \+     Comma         \,
%%   Minus         \-     Point         \.     Solidus       \/
%%   Colon         \:     Semicolon     \;     Less than     \<
%%   Equals        \=     Greater than  \>     Question mark \?
%%   Commercial at \@     Left bracket  \[     Backslash     \\
%%   Right bracket \]     Circumflex    \^     Underscore    \_
%%   Grave accent  \`     Left brace    \{     Vertical bar  \|
%%   Right brace   \}     Tilde         \~}
%%
% \DoNotIndex{\" , \-, \flqq,\frqq, \ ,\def, \begin, \end}
% \DoNotIndex{\csname,\endcsname, \expandafter, \global, \newcommand}
% \DoNotIndex{\advance, \blindfiledate, \blindfileversion, \MessageBreak}
% \DoNotIndex{\newcount, \newcounter, \newif, \or}
% \DoNotIndex{\if, \else, \fi, \ifcase, \ifdefined, \ifnum, \ifx, \loop, \repeat}
% \DoNotIndex{\heartsuit}
% \DoNotIndex{\NeedsTeXFormat, \providecommand, \ProvidesPackage, \relax, \RequirePackage}
% \DoNotIndex{\par, \chapter, \section, \subsection, \subsubsection, \paragraph, }
% \DoNotIndex{\selectlanguage, \stepcounter, \ss, \space, \setcounter}
% \DoNotIndex{\value, \typeout, \long, \renewcommand}
%
% \iffalse
%</package>
%<*driver>
\documentclass{ltxdoc}
\usepackage[ngerman,english]{babel}
\usepackage{blindtext}
\usepackage{makeidx}
\usepackage{varioref}
\IfFileExists{hyperref.sty}{\usepackage{hyperref}}{}
\CodelineIndex
\makeindex
%\OnlyDescription
\begin{document}
 \RecordChanges
 \DocInput{blindtext.dtx}
\end{document}
%</driver>
%
%<*package>
% \fi
%
% \newcommand*{\Lpack}[1]{\textsf {#1}}           ^^A typeset a package
% \newcommand*{\Lenv}[1]{\texttt {#1}}           ^^A typeset a package
%
% \title{Blindtext.sty:\\
%   Creating Dummy Text\\
%   \em Blindtext erzeugen}
%
% \date{\today, \blindfileversion}
% \author{Knut Lickert\thanks{
%       \url{http://tex.lickert.net/packages/blindtext/}}
% }
% \maketitle
%
%\begin{abstract}
%\selectlanguage{english}
%With this package you can create dummy text.
%Use \cmd{\blindtext} to get some text and
%\cmd{\Blindtext} to get a long text.
%With \cmd{\blinddocument} (or \cmd{\Blinddocument})
%you get complete dummy documents.
%
%Details can be found at \url{http://tex.lickert.net/packages/blindtext/index_en.html}.
%This File describes version \blindfileversion.
%
%\selectlanguage{ngerman}
%\em
%Mit diesem Paket kann man lange Texte erzeugen, ohne lange
%Texte einzugeben.
%Sinn ist die Erzeugung langer Beispieltexte, ohne den
%Quelltext lang zu machen.
%
%Details unter \url{http://tex.lickert.net/packages/blindtext/}.
%Dieses Dokument beschreibt Version \blindfileversion.
%\end{abstract}
%
% \tableofcontents
%
%\selectlanguage{english}
%\section{Overview/{\"U}bersicht}
%
%\begin{table}
%\renewcommand{\arraystretch}{1.1}
%\begin{tabular}{p{0.3\textwidth}*{2}{p{0.3\textwidth}}}\hline
%Command/Makro       & Englisch      & German    \\\hline
%\cmd{\blinddocument}   & create a document  & erzeugt ein Dokument\\
%\cmd{\Blinddocument}   & create a big document  & erzeugt ein gro{\ss}es Dokument\\
%\cmd{\blindtext}   & create text  & erzeugt Text\\
%\cmd{\Blindtext}   & create more text & erzeugt mehr Text\\
%\verb|\blindlist{env}|   & create a list  & erzeugt eine Liste\\
%\verb|\blindlistoptional|\newline\hspace*{2em}\verb|{env}|   & create a list with \verb|\item[]|  & erzeugt eine Liste mit \verb|\item[]|\\
%\verb|\blindlistlist|\newline\hspace*{2em}\verb|[level]{env}|   & create cascaded lists  & erzeugt geschachtelte Listen\\
%\verb|\Blindlist{env}|   & create a big list  & erzeugt eine gro{\ss}e Liste\\
%\verb|\Blindlistoptional|\newline\hspace*{2em}\verb|{env}|   & create a big list with \verb|\item[]|  & erzeugt eine gro{\ss}e Liste mit \verb|\item[]|\\
%\verb|\blinditemize|   & create an itemize list  & erzeugt eine itemize-Liste\\
%\verb|\blindenumerate|   & create an enumerate list  & erzeugt eine enume\-ra\-te-Liste\\
%\verb|\blinddescription|   & create a description list  & erzeugt eine des\-crip\-tion-Liste\\\hline
%$\sim$[x]   & Repetition & Wiederholungsfaktor\\
%\end{tabular}
%\caption{Command Overview/Kommando\"ubersicht}
%\end{table}
%
% \subsection{Create Documents}
% \DescribeMacro{\blinddocument}
% \cmd{\blinddocument} create a document with sections,
% subsections\ldots and lists (\Lenv{itemize},
% \Lenv{enumerate} and \Lenv{description}).
%
% \DescribeMacro{\Blinddocument}
% \cmd{\Blinddocument} create a
% document with bigger sections, subsections\ldots and longer
% lists.\par
%
% The smallest document to check the look of a class would be:
% \begin{quote}
% \begin{verbatim}
% \documentclass{<class>}
% \usepackage{blindtext}
% \begin{document}
%     \blinddocument
% \end{document}
% \end{verbatim}
% \end{quote}
%
% If you want to have a table of contents you have two ways:
% \begin{enumerate}
%    \item use the package option ``toc''
%    \item Just add the |\tableofcontents|-command in your testdocument.
%  \end{enumerate}
%
% \subsection{Package options}
% \subsubsection{Math Option}
% The \emph{math}-option activates math formula for |\blindtext|.
%
%  More see in section~\vref{sec:math} and \vref{sec:optionmath}.
%
% \subsubsection{Toc Option}
% Activate the table of contents for |\blinddoccument|.
%
% \subsubsection{Random Option}
% \label{sec:optionrandom}
% \changes{V2.0}{2011-12-28}{Add option random}
% The \emph{random}-option changes the default blind text to 
% a sequence of predefined sentences.
% The next paragraph starts with the next phrase from the previous paragraph.
%
% You may redefine the number of phrases per paragraph,
% details see table~\vref{tab:repetion}.
%
% \begin{table}
% \begin{minipage}{\textwidth}
% \begin{tabular}{llll}
%   Command           & Parameter(s) & Remark & Section\\\hline
%   \cmd{\blindtext} & \oarg{n}  & n repetition\textsuperscript{*}  &\ref{sec:blindtext}\\
%   \cmd{\Blindtext} & \oarg{x}\oarg{n} & x paragraphs with n repetitions&\ref{sec:blindtext}\\[1ex]
%   \multicolumn{3}{l}{\textbf{Global Redefinitions of Default} -- \cmd{\setcounter}\marg{Counter}\marg{n}}\\
%   Counter           & Parameter & Remark & Section\\\hline
%   blindtext          & \marg{n}  & n repetitions per paragraph\\
%   Blindtext          & \marg{n}  & n paragraph for \cmd{\Blindtext}\\
%   blindlist            & \marg{n}  & n items in list&\ref{sec:blindlists}\\
%   blindlistlevel      & \marg{n}  & depth of nested lists (max. 4)&\ref{sec:def:blindlists}\\
% \end{tabular}
% \footnotetext{* The default for \cmd{\blindtext} values depends on selected option
%     (\emph{random}: 17~sentences, \emph{pangram}:~5 pangrams, default:~1 text).}
% \end{minipage}
% \caption{Modifiying Repetion Factors}
% \label{tab:repetion}
% \end{table}
%
% If the language does not support this option, the default text is used.
%
% \subsubsection{Pangram Option}
% \label{sec:optionpangram}
% \changes{V2.0}{2012-01-02}{Add option pangram}
% The \emph{pangram}-option changes the default blind text to 
% a sequence of pangrams.
% A pangram, or holoalphabetic sentence, is a sentence using 
% every letter of the alphabet at least once.\footnote{\url{http://en.wikipedia.org/wiki/Pangram}}
%
% You may redefine the number of phrases per paragraph,
% details see table~\vref{tab:repetion}.
%
% If the language does not support this option, the default text is used.
%
% \subsubsection{Bible Option}
% \label{sec:optionbible}
% \changes{V2.0}{2011-12-27}{Add option bible}
% The \emph{bible}-option changes the default blind text to 
% texts from bible verse 3.14 (Genesis, Exodus, Leviticus, Numeri and Deuteronomium)
%
% \subsection{Get Some Text}
% \label{sec:blindtext}
% \DescribeMacro{\blindtext}
% \cmd{\blindtext}\oarg{x} create a text. The optional parameter define a
% repetition. Default for the repetition is one and can be modified
% with \cmd{\setcounter}\marg{blindtext} \marg{1}.
% See also table~\vref{tab:repetion}.
%
% \DescribeMacro{\Blindtext}
% \cmd{\Blindtext}\oarg{x}\oarg{y} create $x$ paragraphs with the text of
% \cmd{\blindtext}. The first optional parameter defines a
% repetition of the paragraphs. Default for the repetition is five
% and can be modified with \cmd{\setcounter}\marg{Blindtext}\marg{5}. The
% second optional parameter is given to \cmd{\blindtext}.
% See also table~\vref{tab:repetion}.
%
% \DescribeMacro{\parstart}
% \DescribeMacro{\parend}
% It is possible to add a start and end sequence for each paragraph.
%
% The two macros \cmd{\parstart} and \cmd{\parend} define a
% starting and ending sequence for each paragraph in
% \cmd{\Blindtext}.
%
% \changes{V2.0}{2011-12-09}{Default paragraph start}
% The paragraph start is redefined for each language change:
% First paragraphs get an empty start, the following paragraphs get
% different starting texts. 
% Advantage: Each paragraph in a paragraph sequence looks different.
% This feature must be supported by the language\footnote{
%   see redefinitions of \cmd{\blindtext@parstart}}
%
% \subsection{Get Some Lists}
% \label{sec:blindlists}
% \DescribeMacro{\blindlist}
% \cmd{\blindlist}\marg{env}\oarg{x} create a list, the type is defined by the
% obligatory parameter. The optional parameter defines a repetition.
% Default for the repetition is five and can be modified with
% \cmd{\setcounter}\marg{blindlist}\marg{x}.
%
% \DescribeMacro{\blindlistlist}
%  \cmd{\blindlistlist}\oarg{level}\marg{env}\oarg{x} creates cascaded lists up to
%  level \verb|level|, the type is defined by the second obligatory
%  parameter. The optional parameter defines a repetition. Default is the
%  same like in \cmd{\blindlist}.
%
% \DescribeMacro{\Blindlist}
% \cmd{\Blindlist}\marg{env}\oarg{x} create a list with long texts.
% The optional parameter defines a
% repetition. Default for the repetition is five and can be modified
% with \cmd{\setcounter}\marg{blindlist}\marg{x}.
%
% \DescribeMacro{\blindlistoptional}
% \cmd{\blindlistoptional}\marg{env}\oarg{x} create a list with \cmd{\item}\oarg{}. The
% list is defined by the obligatory parameter. The optional
% parameter defines a repetition. Default for the repetition is five
% and can be modified with \cmd{\setcounter}\marg{blindlist} \marg{x}.
%
% \DescribeMacro{\Blindlistoptional}
% \cmd{\Blindlistoptional} create a list like
% \cmd{\blindlistoptional} with long texts.
%
% \DescribeMacro{\blinditemize}
% \DescribeMacro{\blindenumerate}
% \DescribeMacro{\blinddescription}
% The commands \cmd{\blinditemize}, \cmd{\blindenumerate} and \cmd{\blinddescription}
% are abbreviations for \cmd{\blindlist}\marg{itemize}, \cmd{\blindlist}\marg{enumerate}
% and \cmd{\blindlist}\marg{description}.
%
% \DescribeMacro{\Blinditemize}
% \DescribeMacro{\Blindenumerate}
% \DescribeMacro{\Blinddescription}
% The commands \cmd{\Blinditemize}, \cmd{\Blindenumerate} and \cmd{\Blinddescription}
% are abbreviations for \cmd{\Blindlist}\marg{itemize}, \cmd{\Blindlist}\marg{enumerate}
% and \cmd{\Blindlist}\marg{description}.
%
%
% \subsection{Get Some Math in the Text}
% \label{sec:math}
% \marginpar{\hfill Option math}
% If you need a dummy text with math formula, you can add
% some math with the package option ``math''.
%
%\DescribeMacro{\blindmathpaper}
% |\blindmathpaper| built some text with formulas in between.
% This macro is used in |\blinddocument| if |\blindmathtrue| is set.
%
% If you need dummy text with and without math, you can (de)activate
% the math option with the following commands:
%
%\DescribeMacro{\blindmathtrue}
% With |\blindmathtrue| you set a flag, so the following blind text contains
% inline math.
%
%\DescribeMacro{\blindmathfalse}
% With |\blindmathfalse| you reset the flag for inline math
% inside the blind text.
%
% \subsection{Get Additional Markup in the Text}
% \DescribeMacro{\blindmarkup}
% If you want to test some markup in your dummy text you can redefine 
% |\blindmarkup|.
%
% |\blindmarkup| is a macro with one parameter and it set some 
% parts in your markup.
%
% Example:
% \begin{verbatim}
%   \renewcommand{\blindmarkup}[1]{\emph{#1}}
% \end{verbatim}
% 
% \section{FAQ/More features}
% 
% \subsection{Add Tabulars and Tables}
% Is it possible to add tabulars to blindtext?
%
% There are no plans to integrate tabulars in blindtext.
% Tabulars in text are no good idea, they should be integrated
% to tables-environment (tabular, may be a short description, caption).
%
% Tabulars need there own layout, I don't think you may define
% a table for each use in blindtext.
%
% You may define your own tabular/table and surround it with
% |\blindtext|.
%
% Example:
% \begin{verbatim}
% \Blindtext
% % Your tabular
% \Blindtext
% \end{verbatim}
%
% \subsection{Add Pictures and Figures }
% Is it possible to add pictures to blindtext?
%
% There are no plans to integrate pictures in blindtext.
%
% I don't think you may define
% a picture for each use in blindtext.
%
%
% \subsection{Key-Value-Options}
%
% You may change some defaults by redefining commands\footnote{macro \cmd{\blindmarkup}}
%        or numbers\footnote{|blindtext|, |Blindtext|, |blindlist| and |blindlistlevel|}.
% Would it be possible to define them as key-value options?
%
% In theory yes, but I don't want to add an additional dependency to other packages.
%
%
% \subsection{More Languages}
% The language xx is not supported - can you add it?
%
% If you want to have blindtext in another language, please provide me some texts.
%
% I need:
% \begin{itemize}
%  \item The language name (used in babel)
%  \item Some text for the paragraphs
%  \item Some (at least three) sentences for paragraph start.
%  \item The numbers as used in ``first, second\ldots''
%  \item Some sentences to be used with option \emph{random}.
%          You may tell a little story and the last sentence can be continued with 
%           the first sentence.
%  \item Some pangrams (See \url{http://en.wikipedia.org/wiki/List_of_pangrams}.
%  \item The following texts:
%  \begin{itemize}
%    \item item: ``item in a list''
%    \item heading: ``Heading on level''
%    \item lists: ``Lists''
%    \item listsEx: ``Example for list''
%    \item blindmath: ``Some blind text with math formulas''
%  \end{itemize}
% \end{itemize}

% \appendix
% \StopEventually
%
% \selectlanguage{english}
% \section{Implementation}
%    \changes{V1.8}{2009-01-27}{Adding a lot percent to avoid spaces.}
%
%
%    \begin{macrocode}
\NeedsTeXFormat{LaTeX2e}
\ProvidesPackage{blindtext}[\blindfiledate\space\blindfileversion\space%
                 blindtext-Package]
\RequirePackage{xspace}
%    \end{macrocode}
%    \begin{macro}{\grqq}
%    \begin{macro}{\glqq}
% Just in case the German quote are not defined
%    \begin{macrocode}
\providecommand{\grqq}{}
\providecommand{\glqq}{}
%    \end{macrocode}
%    \end{macro}
%    \end{macro}
%
%    \begin{macro}{\blind@checklanguage}
%    \changes{V1.6}{2006-08-02}{Check for defined language}
%    \changes{V2.0}{2011-12-31}{Text default lorem lipsum}
%    It is possible, that you use blindtext.sty with
%    undefined languages.
%    This macro checks if blindtext supports the language.
%    If it is not available, an error is reported and
%    the default lorem lipsum used.
%
%    If you don't load babel, the message may be confusing:
%    I get the warning ``welsh not defined''.\footnote{
%             \LaTeX\ loads babel on it's own to load hyphenations,
%             Welsch is the last language in alphabet.
%             }
%    \begin{macrocode}
\newcommand{\blind@checklanguage}{%
  \@ifundefined{blindtext@\languagename}{%
    \PackageWarning{blindtext}{\languagename\space not
      defined, using English instead.\MessageBreak
    }%
  }{}%
}
%    \end{macrocode}
%    \end{macro}
%
%
% \subsection{Counter and Supporting Macros}
% First we need some counters for the number of repetitions of the
% text and the paragraphs.
%    \begin{macrocode}
\newcounter{blindtext}\setcounter{blindtext}{1}
\newcounter{Blindtext}\setcounter{Blindtext}{5}
%    \end{macrocode}
%
% Define a counter for paragraph start sequence.
%    \begin{macrocode}
\newcounter{blind@countparstart}
%    \end{macrocode}
%
%    \begin{macro}{\blindtext}
%
% \verb|\blindtext[x]| writes a sentence x times.
% The default is stored in the counter \verb|blindtext|. This value
% can be changed with \verb|\setcounter{blindtext}{1}|.
%
%    \changes{V1.6}{2006-08-02}{Check for defined language}
%    \begin{macrocode}
\newcount\blind@countxx
\newcommand{\blindtext}[1][\value{blindtext}]{%
  \blind@checklanguage
  \setcounter{blind@randommax}{#1}%for option random
  \setcounter{blind@pangrammax}{#1}%for option pangram
  \blind@countxx=1 %
  \loop
    \blindtext@text\
  \ifnum\blind@countxx<#1\advance\blind@countxx by 1 %
  \repeat
}
%    \end{macrocode}
%    \end{macro}
%
% \subsection{Creating Text}
%    \begin{macro}{\Blindtext}
% \verb|\Blindtext[x][y]| execute $x\times$\verb|\blindtext[y]|. Each
% \verb|\blindtext[y]| built a paragraph.
% The default is stored in the counter \verb|Blindtext|. This value
% can be changed with \verb|\setcounter{Blindtext}{5}|.
% The counter \verb|blindtext@numBlindtext| stores the first
% optional parameter to be used in \verb|\blindtext@Blindtext|.
%    \changes{V1.6}{2006-08-02}{Check for defined language}
%    \begin{macrocode}
\newcount\blindtext@numBlindtext
\newcommand{\Blindtext}[1][\value{Blindtext}]{%
  \blind@checklanguage
  \blindtext@numBlindtext=#1\relax
  \blind@Blindtext
}
%    \end{macrocode}
%    \end{macro} %^^A Blindtext
%
% \begin{macro}{\blind@Blindtext}
% This macro continue \cmd{\Blindtext} with a second optional parameter.
%
% Each paragraph starts and end with a text, which can be defined with
% \verb|\starttext| or \verb|\endtext|.
%    \begin{macrocode}
\newcount\blind@countyy
\newcommand{\blind@Blindtext}[1][\value{blindtext}]{%
  \blind@countyy=1 %
  \loop
    {\blindtext@parstart\blindtext[{#1}]\blindtext@parend\par}%
  \ifnum\blind@countyy<\blindtext@numBlindtext\advance\blind@countyy by 1 %
  \repeat
}
%    \end{macrocode}
% \end{macro}%^^A {\blind@Blindtext}
%
% \begin{macro}{\parstart}
% \begin{macro}{\parend}
%    \begin{macrocode}
\newcommand{\blindtext@parstart}{}   % Text at start of paragraph
\newcommand{\blindtext@parend}{}     % Text at end   of paragraph
\newcommand{\parstart}[1]{\renewcommand{\blindtext@parstart}{#1}}
\newcommand{\parend}[1]{\renewcommand{\blindtext@parend}{#1}}
%    \end{macrocode}
% \end{macro}%^^A{\parstart}
% \end{macro}%^^A{\parend}
%
% \subsection{Lists}
% \label{sec:def:blindlists}
% Here I define some flags to decide in the lists if we have to
% create big items and if we need the optional parameter at
% \verb|\item|. These flags are reset in \verb|\blind@list|.\par
% The counter \verb|blindlist| define the default number of items in a list.
% The counter \verb|blindlistitem| is needed by \verb|\blindtext@count|
% to write a text like 'first', 'second'\ldots.
%    \begin{macrocode}
\newif\ifblind@long\blind@longfalse
\newif\ifblind@optional\blind@optionalfalse
\newcounter{blindlist}
\newcounter{blindlistlevel}% Up tu X level
\newcounter{blindlist@level}% internal counter
\newcount\blind@listitem
%    \end{macrocode}
%
% \begin{macro}{\blindlist}
% \begin{macro}{\blind@listtype}
% \verb|\blindlist{env}[x]| writes an env-list with $x$ items.
% The default is stored in the counter \verb|blindlist|. This value
% can be changed with \verb|\setcounter{blindlist}{1}|.
% The type of list is stored in \verb|\blind@listtype|.
%    \changes{V1.6}{2006-08-02}{Check for defined language}
%    \begin{macrocode}
%% ^^A-----------------------------------------------------------
\newcommand{\blindlist}[1]{%
  \blind@checklanguage
  \def\blind@listtype{#1}%
  \setcounter{blind@levelcount}{1}%
  \blind@list
}
%    \end{macrocode}
% \end{macro}%^^A{\blindlist}
% \end{macro}%^^A{\blind@listtype}
%
% \begin{macro}{\blind@list}
% This macro continue the macro \verb|\blindlist| and with the
% optional parameter. After the printout, we reset the flags and the
% counter for the list text.
%    \begin{macrocode}
\newcommand{\blind@list}[1][5]{%
  \setcounter{blindlist}{#1}%
  \stepcounter{blindlist@level}% depth of list
  \blind@listitem=1 %
  \begin{\blind@listtype}%
    \blind@items
  \end{\blind@listtype}%
  \blind@longfalse
  \blind@optionalfalse
}
%    \end{macrocode}
% \end{macro}%^^A {\blind@list}
%
% \begin{macro}{\blindlistlist}
% \verb|\blindlistlist[num]{env}[x]| writes cascaded lists up to level num.
%    \begin{macrocode}
\newcommand{\blindlistlist}[2][4]{%
  \setcounter{blindlistlevel}{#1}%
  \setcounter{blindlist@level}{0}% deepest reached level
  \setcounter{blind@levelcount}{1}% actual level for printout
  \def\blind@listtype{#2}%
  \blind@list
}
%    \end{macrocode}
% \end{macro}%^^A {\blindlistlist}
% \begin{macro}{\blindlistlistoptional}
% Like \verb|\blindlistlist[num]{env}[x]| but for environment with optional parameters.
%    \begin{macrocode}
\newcommand{\blindlistlistoptional}{%
  \blind@optionaltrue
  \blindlistlist
}
%    \end{macrocode}
% \end{macro}%^^A{\blindlistlistoptional}
%
% \begin{macro}{\blinditemize}
% \begin{macro}{\blindenumerate}
% Some default macros for the standard environments.
%    \begin{macrocode}
\newcommand{\blinditemize}{\blindlist{itemize}}
\newcommand{\blindenumerate}{\blindlist{enumerate}}
%    \end{macrocode}
% \end{macro}%^^A{\blinditemize}
% \end{macro}%^^A{\blindenumerate}
%
% \verb|\Blindlist[x]| write a list with $x$ items.
% The default is stored in the counter \verb|blindlist|. This value
% can be changed with \verb|\setcounter{blindlist}{1}|.
%
%
% \begin{macro}{\Blindlist}
%    \begin{macrocode}
\newcommand{\Blindlist}[1]{%
  \blind@longtrue
  \blindlist{#1}%
}
%    \end{macrocode}
% \end{macro}%^^A{\Blindlist}
%
% \begin{macro}{\Blinditemize}
% \begin{macro}{\Blindenumerate}
% Some default macros for the standard environments.
%    \begin{macrocode}
\newcommand{\Blinditemize}{\Blindlist{itemize}}
\newcommand{\Blindenumerate}{\Blindlist{enumerate}}
%    \end{macrocode}
% \end{macro}%^^A{\Blinditemize}
% \end{macro}%^^A{\Blindenumerate}
%
% \begin{macro}{\blindlistoptional}
% Here we start a list with \verb|\item[]|. So we set the flag for optional
% parameter and start the normal list.
%    \begin{macrocode}
\newcommand{\blindlistoptional}[1]{%
  \blind@optionaltrue
  \blindlist{#1}%
}
%    \end{macrocode}
% \end{macro}%^^A{\blindlistoptional}
%
% \begin{macro}{\Blindlistoptional}
% Now a big list with optional parameter at \verb|\item|.
%    \begin{macrocode}
\newcommand{\Blindlistoptional}[1]{%
  \blind@optionaltrue
  \blind@longtrue
  \blindlist{#1}%
}
%    \end{macrocode}
% \end{macro}%^^A{\blindlistoptional}
%
% \begin{macro}{\blinddescription}
% \begin{macro}{\Blinddescription}
% Some default macros for the standard environment description.
%\changes{1.7}{2006-12-01}{Replace counter listcount}
%    \begin{macrocode}
\newcommand{\blinddescription}{\blindlistoptional{description}}
\newcommand{\Blinddescription}{\Blindlistoptional{description}}
\newcounter{blind@listcount}
\newcounter{blind@levelcount}
%    \end{macrocode}
% \end{macro}%^^A{\blinddescription}
% \end{macro}%^^A{\Blinddescription}
%
% \begin{macro}{\blind@items}
% Here the \verb|\item| are written.
% A local counter is defined and the \verb|\item| is written. The
% type of \verb|\item| is influenced by the flags.
%    \begin{macrocode}
\newcommand{\blind@items}{%
  \setcounter{blind@listcount}{1}%
  \loop
  \ifblind@optional
    \ifblind@long
      \item[\blindtext@countitem] \blindtext@text
    \else
      \item[\blindtext@count] \blindtext@item
    \fi
    \else % \blind@optionalfalse
      \ifblind@long
        \item \blindtext@text
      \else
        \item \blindtext@countitem
      \fi
    \fi % \ifblind@optional
%    \end{macrocode}
% Loop for cascaded lists.
% \changes{V1.3}{2003-05-18}{Cascaded lists}
%    \begin{macrocode}
    {%
      \loop
      \ifnum\value{blindlistlevel}>\value{blindlist@level}%
        \stepcounter{blind@levelcount}%
        \blind@list[\value{blindlist}]\relax
        \addtocounter{blind@levelcount}{-1}%
        \setcounter{blind@listcount}{1}%
      \repeat
    }%
  \ifnum\value{blind@listcount}<\value{blindlist}%
    \stepcounter{blind@listcount}%
%    \end{macrocode}
% Correction for nested values.
%    \begin{macrocode}
%    \end{macrocode}
% Close the loop
%    \begin{macrocode}
  \repeat
}
%    \end{macrocode}
% \end{macro}%^^A{\blind@item}
%% ^^A-----------------------------------------------------------
% \subsection{Blind Text with Markup}
%    \changes{V1.9d}{2010-01-14}{Adding markup feature}
%    \begin{macro}{\blindmarkup}
%    \begin{macrocode}
\newcommand{\blindmarkup}[1]{#1}
%    \end{macrocode}
%    \end{macro}% ^^A \blindmarkup
%
%% ^^A-----------------------------------------------------------
% \subsection{Blind Text with Math}
%    \changes{V1.9}{2009-05-06}{Adding math}
% \subsubsection{Inline Math}
%    \begin{macro}{\blindtext@endsentence}
% Finish a sentence with a dot.
% This macro is redefined for blind text with inline math/formulas.
%    \begin{macrocode}
  \def\blindtext@endsentence{.\xspace}%
%    \end{macrocode}
%
% Counter to make some alternation of inline formulas.
%    \begin{macrocode}
  \newcount{\blind@mathformula}\blind@mathformula=0%
%    \end{macrocode}
%
% A flag to sign, if math formulas should be used in blind text.
%    \begin{macrocode}
\newif\ifblindmath
%    \end{macrocode}
%
%    \begin{macro}{\blindmathtrue}
% Make the following blind text with inline math.
% This redefine the logical variable |\ifblindmath|,
% to keep the flag we have to store and call the old definition.
%    \begin{macrocode}
\let\oldblindmathtrue\blindmathtrue
\renewcommand{\blindmathtrue}{
    \oldblindmathtrue
%    \end{macrocode}
% Make a formula each x sentence.
%    \begin{macrocode}
  \def\blindtext@endsentence{%
    \advance\blind@mathformula by 1%
    \ifcase\blind@mathformula%
    \or. \(\sin^2(\alpha) + \cos^2(\beta) = 1\)%
    \or\xspace\(E = mc^2\)%
    \or. \(\sqrt[n]{a} \cdot \sqrt[n]{b} =  \sqrt[n]{ab}\)%
    \or. \(\frac{\sqrt[n]{a}}{\sqrt[n]{b}} =  \sqrt[n]{\frac{a}{b}}\)%
    \or. \(a\sqrt[n]{b} = \sqrt[n]{a^n b}\)%
    \or. \(\mathrm{d}\Omega = \sin \vartheta \mathrm{d} \vartheta \mathrm{d}\varphi\)%
    \else\global\blind@mathformula=0%
    \fi%
    .\xspace}%
%    \end{macrocode}
%    \begin{macrocode}
  }%\blindmathtrue
%    \end{macrocode}
%    \end{macro}    %^^A \blindmathtrue
%
%    \begin{macro}{\blindmathfalse}
% Make the following blind text without inline math.
%    \begin{macrocode}
\let\oldblindmathfalse\blindmathfalse
\renewcommand{\blindmathfalse}{
    \oldblindmathfalse
%    \end{macrocode}
% Rebuild the normal sentence end.
%    \begin{macrocode}
  \def\blindtext@endsentence{.\xspace}%
%    \end{macrocode}
%    \begin{macrocode}
}%\includemath
%    \end{macrocode}
%    \end{macro}    %^^A \blindmathfalse
%    \end{macro}    %^^A\blindtext@endsentence
%
%
% \subsubsection{'Big' Formulas}
% Counter to make some alternation of 'big' formulas.
%    \begin{macrocode}
  \newcount{\blind@Mathformula}
  \blind@Mathformula=0%
%    \end{macrocode}
%    \begin{macro}{\blindtext@formula}
% Make a formula each x sentence.
%    \begin{macrocode}
  \def\blindtext@formula{%
    \advance\blind@Mathformula by 1%
    \ifcase\blind@Mathformula%
        \[\bar x = \frac{1}{n}\sum_{i=1}^{i=n} x_i = \frac{x_1 + x_2 + \dots{} + x_n}{n}\]
    \or \[ \int_0^\infty e^{-\alpha x^2} \mathrm{d}x =
            \frac12\sqrt{\int_{-\infty}^\infty e^{-\alpha x^2}}
            \mathrm{d}x\int_{-\infty}^\infty e^{-\alpha y^2}\mathrm{d}y =
            \frac12\sqrt{\frac{\pi}{\alpha}} \]
    \or \[ \sum_{k=0}^\infty a_0q^k = \lim_{n\to\infty}\sum_{k=0}^n a_0q^k =
            \lim_{n\to\infty} a_0\frac{1-q^{n+1}}{1-q} = \frac{a_0}{1-q}
        \]
    \or \[x_{1,2}=\frac{-b \pm \sqrt{b^2-4ac}}{2a} = \frac{-p \pm \sqrt{p^2-4q}}{2}\]
    \or \[ \frac{\partial^2 \Phi}{\partial x^2} + \frac{\partial^2 \Phi}{\partial y^2} +
            \frac{\partial^2 \Phi}{\partial z^2} =
            \frac{1}{c^2}\frac{\partial^2\Phi}{\partial t^2}
        \]
    \or \[\sqrt[n]{a} \cdot \sqrt[n]{b} =  \sqrt[n]{ab}\]
    \or \[\frac{\sqrt[n]{a}}{\sqrt[n]{b}} =  \sqrt[n]{\frac{a}{b}}\]
    \or \[a\sqrt[n]{b} = \sqrt[n]{a^n b}\]
    \global\blind@Mathformula=0%
    \fi%
}%
%    \end{macrocode}    %^^A \blindtext@formula
%    \end{macro}    %^^A \blindtext@formula
%
%% ^^A-----------------------------------------------------------
% \subsection{Create Complete Documents}
% A flag to sign, if the dummy documents should include a
% table of contents.
%    \begin{macrocode}
\newif\ifblindtoc
\blindtocfalse
%    \end{macrocode}
%
%    \begin{macro}{\blinddocument}
% \cmd{\blinddocument} adds a document with \verb|\chapter|,
% \verb|\section|\ldots, texts and lists (itemize, enumerate,
% description).
%    \begin{macrocode}
\newcommand{\blinddocument}{%
  \ifblindtoc\tableofcontents\fi
  \@ifundefined{chapter}{}{%
    \chapter{\blindtext@heading 0 (chapter)}%
    \blindtext
  }%
  \section{\blindtext@heading 1 (section)}%
    \blindtext
  \subsection{\blindtext@heading 2 (subsection)}%
    \blindtext
  \subsubsection{\blindtext@heading 3 (subsubsection)}%
    \blindtext
  \paragraph{\blindtext@heading 4 (paragraph)}%
    \blindtext
  \section{\blindtext@list}%
  \subsection{\blindtext@listEx (itemize)}%
    \blinditemize
  \subsubsection{\blindtext@listEx (4*itemize)}%
    \blindlistlist{itemize}[2]%
  \subsection{\blindtext@listEx (enumerate)}%
    \blindenumerate
  \subsubsection{\blindtext@listEx (4*enumerate)}%
    \blindlistlist{enumerate}[2]%
  \subsection{\blindtext@listEx (description)}%
    \blinddescription
  \subsubsection{\blindtext@listEx (4*description)}%
    \blindlistlistoptional{description}[2]%
}
%    \end{macrocode}
%    \end{macro}%^^A{\blinddocument}
%
%    \begin{macro}{\Blinddocument}
% \cmd{\blinddocument} adds a document with \verb|\chapter|,
% \verb|\section|\ldots, texts and lists (itemize, enumerate,
% description).
%    \begin{macrocode}
\newcommand{\Blinddocument}{%
  \ifblindtoc\tableofcontents\fi
  \@ifundefined{chapter}{}{%
    \chapter{\blindtext@heading 0 (chapter)}%
    \blindtext
  }%
  \section{\blindtext@heading 1 (section)}%
    \Blindtext
  \subsection{\blindtext@heading 2 (subsection)}%
    \Blindtext
  \subsubsection{\blindtext@heading 3 (subsection)}%
    \Blindtext
  \paragraph{\blindtext@heading 4 (paragraph)}%
    \Blindtext
  \section{\blindtext@list}%
  \subsection{\blindtext@listEx (itemize)}%
    \Blinditemize
  \subsubsection{\blindtext@listEx (4*itemize)}%
    \blind@longtrue
    \blindlistlist{itemize}[2]%
  \subsection{\blindtext@listEx (enumerate)}%
    \Blindenumerate
  \subsubsection{\blindtext@listEx (4*enumerate)}%
    \blind@longtrue
    \blindlistlist{enumerate}[2]%
  \subsection{\blindtext@listEx (description)}%
    \Blinddescription
  \subsubsection{\blindtext@listEx (4*description)}%
    \blind@longtrue
  \blindlistlistoptional{description}[2]%
  %
  \ifblindmath
      \section{\blindtext@blindmath}%
      \blindmathpaper%
  \fi%
}
%    \end{macrocode}
%    \end{macro}%^^A{\Blinddocument}
%
%    \begin{macro}{\blindmathpaper}
%    \changes{V1.9}{2009-05-06}{Adding math}
% Build some text with formulas in between.
%    \begin{macrocode}
\newcommand{\blindmathpaper}{
\blindtext
\blindtext@formula
\blindtext
\blindtext@formula
\blindtext
\blindtext@formula
\blindtext
\blindtext@formula
\blindtext
\blindtext@formula
\blindtext\relax%
}%\blindmathpaper
%    \end{macrocode}
%    \end{macro} %^^A \blindmathpaper
%
%
% \section{Option Processing}
% \subsection{Bible Option}
%    \label{sec:optionbible}
% \subsection{Bible Option}
%    \changes{V2.0}{2011-12-27}{Option bible}
%
% The option bible change the default blind text to texts from the bible.
%    \begin{macrocode}
\newif\ifblindbible
\DeclareOption{bible}{
  \blindbibletrue
}
%    \end{macrocode}
%
%
% \subsection{Random Option}
%    \changes{V2.0}{2011-12-27}{Option random}
%    \begin{macrocode}
\newif\ifblindrandom
%    \end{macrocode}
%    Define a counter for continues text change.
%    \begin{macrocode}
\newcounter{blind@randomcount}\setcounter{blind@randomcount}{0}
%    \end{macrocode}
%  Define a counter to define the number of sentences per paragraph when you use the random option.
%  blind@randommax is redefined in \cmd{blindtext}
%    \begin{macrocode}
\newcounter{blind@randommax}
%    \end{macrocode}
%
% The option random change the default blind text to texts from the random.
% Changes also the counter \emph{blindtext} (Default parameter for \cmd{blindtext}).
%    \begin{macrocode}
\DeclareOption{random}{
\blindrandomtrue
}
%    \end{macrocode}
%
%
% \subsection{Pangram Option}
%    \changes{V2.0}{2012-01-02}{Option pangram}
%    \begin{macrocode}
\newif\ifblindpangram
%    \end{macrocode}
%    Define a counter for continues text change.
%    \begin{macrocode}
\newcounter{blind@pangramcount}\setcounter{blind@pangramcount}{0}
%    \end{macrocode}
%  Define a counter to define the number of sentences per paragraph when you use the pangram option.
%  blind@pangrammax is redefined in \cmd{blindtext}
%    \begin{macrocode}
\newcounter{blind@pangrammax}
%    \end{macrocode}
%
% The option pangram change the default blind text to texts from the pangram.
% Changes also the counter \emph{blindtext} (Default parameter for \cmd{blindtext}).
%    \begin{macrocode}
\DeclareOption{pangram}{
\blindpangramtrue
}
%    \end{macrocode}
%
%
% \subsection{Math Option}
%    \label{sec:optionmath}
%    \changes{V1.9}{2009-06-06}{Option math}
% Activate the math formulas in the text.
%    \begin{macrocode}
\DeclareOption{math}{
    \blindmathtrue
}
%    \end{macrocode}
%
% \subsection{Toc Option}
%    \label{sec:optiontoc}
%    \changes{V1.9}{2009-06-14}{Option toc}
% Activate the table of contents for |\blinddoccument|.
%    \begin{macrocode}
\DeclareOption{toc}{
    \blindtoctrue
}
%    \end{macrocode}
%
% \section{Closing Actions}
% Activate the options
%    \begin{macrocode}
\ProcessOptions\relax
%    \end{macrocode}
%
%
%% ^^A-----------------------------------------------------------
% \section{The Texts}
%
%    \changes{V1.7}{2006-11-21}{Delete trailing spaces in language definitions.}
%
%    \begin{macro}{\blind@addtext}
% This macro adds the texts according to the language definitions.
% Four language packages are supported:
%\begin{itemize}
%  \item babel
%  \item polyglossia (see \url{http://tug.ctan.org/tex-archive/macros/xetex/latex/polyglossia/})
%  \item german
%  \item ngerman
%\end{itemize}
%
%Parameters:
%\begin{enumerate}
%  \item Language
%  \item The text definitions
%\end{enumerate}
%    \begin{macrocode}
\newcommand{\blind@addtext}[2]{%
%    \end{macrocode}
%   First we take care of the babel-package.
%    \begin{macrocode}
    \@ifpackageloaded{babel}{
        \expandafter\addto\csname extras#1\endcsname{#2}
    }{}%
%    \end{macrocode}
%    \changes{V1.9}{2009-06-03}{Support polyglossia}
% Polyglossia provides a complete Babel replacement for users of Xe\LaTeX.
%    \begin{macrocode}
    \@ifpackageloaded{polyglossia}{
        \expandafter\gappto\csname captions#1\endcsname {#2}
    }{}%
%    \end{macrocode}
%    \changes{V1.9}{2009-06-04}{Support (n)german}
% Now some special support for the (n)german-package.
%    \begin{macrocode}
    \@ifpackageloaded{ngerman}{
        \expandafter\g@addto@macro\csname captions#1\endcsname {#2}
    }{}%
    \@ifpackageloaded{german}{
        \expandafter\g@addto@macro\csname captions#1\endcsname {#2}
    }{}%
%    \end{macrocode}
%
%    \begin{macrocode}
}%\blind@addtext
%    \end{macrocode}
%    \end{macro}%^^A\blind@addtext
%
%
% \subsection{Default Without Language}
% First all default texts if no language is selected.
%
%    \begin{macrocode}
\def\blindtext@text{%
    Lorem ipsum dolor sit amet, consectetuer adipiscing elit. Etiam
    lobortis facilisis sem. Nullam nec mi et neque pharetra
    sollicitudin. Praesent imperdiet mi nec ante. Donec ullamcorper,
    felis non sodales commodo, lectus velit ultrices augue, a
    dignissim nibh lectus placerat pede. Vivamus nunc nunc, molestie
    ut, ultricies vel, semper in, velit. Ut porttitor. Praesent in
    sapien. Lorem ipsum dolor sit amet, consectetuer adipiscing elit.
    Duis fringilla tristique neque. Sed interdum libero ut metus.
    Pellentesque placerat. Nam rutrum augue a leo. Morbi sed elit sit
    amet ante lobortis sollicitudin. Praesent blandit blandit mauris.
    Praesent lectus tellus, aliquet aliquam, luctus a, egestas a,
    turpis. Mauris lacinia lorem sit amet ipsum. Nunc quis urna dictum
    turpis accumsan semper.%
}
\def\blindtext@count{%
  \ifcase\blind@listitem\or
    First\or Second%...
  \else
    Last%
    \blind@listitem=0 %
  \fi
  \global\advance\blind@listitem by 1 %
}% \blindtext@count
\def\blindtext@item{itemtext}
\def\blindtext@countitem{\blindtext@count\ \blindtext@item}
\def\blindtext@heading{Heading on level\xspace}
\def\blindtext@list{Lists}
\def\blindtext@listEx{Example for list\xspace}
\def\blindtext@blindmath{Some blind text with math formulas}
%    \end{macrocode}
%</package>
%
% ^^A Add text documentations
% %
%
% ^^A This document is generated by mk_blindtext_texts.rb
%
%
% \subsection{English Texts (babel: english)}
% \changes{V1.9e}{2011-12-09}{Correction English}
% Thanks to Felix Lehmann for corrections.
%
%    \begin{macro}{\blindtext@english}
%    Define flag, so we can check if language is defined.
%    \begin{macrocode}
\def\blindtext@english{}
%    \end{macrocode}
%    \end{macro}
%
% Define the default blind text for English.
%    \begin{macrocode}
\blind@addtext{english}{%
  \def\blindtext@text{%
    Hello, here is some text without a meaning\blindtext@endsentence
    This text should show what a printed text will look like at this
    place\blindtext@endsentence If you read this text, you will get no
    information\blindtext@endsentence Really? Is there no information?
    Is there a difference between this text and some nonsense like
    ``Huardest gefburn''? Kjift -- not at all! A blind text
    \blindmarkup{like this} gives you information  about the selected
    font, how the letters are written and an impression  of the
    look\blindtext@endsentence This text should contain \blindmarkup{all
    letters of the alphabet} and it should be written in of the original
    language\blindtext@endsentence There is no need for  special
    content, but the length of words should match the
    language\blindtext@endsentence%
  }% \blindtext@text
}
%    \end{macrocode}
%
%
% \changes{V1.9e}{2011-12-09}{Default paragraph start for English}
% Define different paragraph starts for second and later paragraphs.
% The first paragraph gets no special start.
%    \begin{macrocode}
\blind@addtext{english}{%
  \def\blindtext@parstart{%
      \ifcase\value{blind@countparstart}\or
This is the second paragraph.\or
And after the second paragraph follows the third paragraph.\or
After this fourth paragraph, we start a new paragraph sequence.\or
        \setcounter{blind@countparstart}{0}
      \fi
      \stepcounter{blind@countparstart}
  }% \blindtext@parstart
}
%    \end{macrocode}
%
% Define counters for list environments.
%    \begin{macrocode}
\blind@addtext{english}{%
  \def\blindtext@count{%
    \ifcase\value{blind@listcount}\or
      First\or Second\or Third\or Fourth\or Fifth\or
      Sixth\or Seventh\or Eighth\or Ninth\or Tenth\or
      Eleventh\or Twelfth%
    \else
      Another%
    \fi
  }% \blindtext@count
  \def\blindtext@item{item in a list}%
}%\addto\extrasenglish
%    \end{macrocode}
%
% Define title lines for English.
%    \begin{macrocode}
\blind@addtext{english}{%
  \def\blindtext@heading{Heading on Level\xspace}%
  \def\blindtext@list{Lists}%
  \def\blindtext@listEx{Example for list\xspace}%
}%\addto\extrasenglish
%    \end{macrocode}
%
% Add the title for |\blindmathpaper|.
%    \begin{macrocode}
\blind@addtext{english}{%
    \def\blindtext@blindmath{Some blind text with math formulas}%
}%\addto\extrasenglish
%    \end{macrocode}
%
%
% Define the bible-option text for english.
%    \begin{macrocode}
\ifblindbible
\blind@addtext{english}{%
  \def\blindtext@text{%
    And the Lord God said unto the serpent, Because thou hast done this,
    thou art cursed above all cattle, and above every beast of the
    field; upon thy belly shalt thou go, and dust shalt thou eat all the
    days of thy life:
    And God said unto Moses, `I am that I am': and he said, Thus shalt
    thou say unto the children of Israel, `I am' hath sent me unto you.
    And he shall offer thereof his offering, even an offering made by
    fire unto the Lord; the fat that covereth the inwards, and all the
    fat that is upon the inwards,\ldots
    And the Lord spake unto Moses in the wilderness of Sinai,
    saying,\ldots
    Jair the son of Manasseh took all the country of Argob unto the
    coasts of Geshuri and Maachathi; and called them after his own name,
    Bashanhavothjair, unto this day.%
  }% \blindtext@text
  \def\blindtext@parstart{}%no change for bible option
}
\fi %\ifbible
%    \end{macrocode}
%
% Define the random-option text for english.
%    \begin{macrocode}
\ifblindrandom
  \PackageWarning{blindtext}{Option random not defined for english\MessageBreak}%
  \blind@addtext{english}{%
    \setcounter{blindtext}{1}
  }
\fi %option random
%    \end{macrocode}
%
% Define the pangram-option text for english.
%    \begin{macrocode}
\ifblindpangram
\blind@addtext{english}{%
    \setcounter{blindtext}{5}
    \def\blindtext@text{%
    \blind@countxx=1 %
    \loop  
      \ifcase\value{blind@pangramcount}%
The quick brown fox jumps over the lazy dog\blindtext@endsentence
\or Jackdaws love my big Sphinx of Quartz\blindtext@endsentence
\or Pack my box with five dozen liquor jugs\blindtext@endsentence
\or The five boxing wizards jump quickly\blindtext@endsentence
\or Sympathizing would fix Quaker objectives\blindtext@endsentence
\or Many-wived Jack laughs at probes of sex quiz\blindtext@endsentence
\or Turgid saxophones blew over Mick's jazzy quaff\blindtext@endsentence
\or Playing jazz vibe chords quickly excites my
wife\blindtext@endsentence
\or A large fawn jumped quickly over white zinc
boxes\blindtext@endsentence
\or Exquisite farm wench gives body jolt to prize
stinker\blindtext@endsentence
\or Jack amazed a few girls by dropping the antique onyx vase!\xspace%
    \setcounter{blind@pangramcount}{-1}%
    \fi%
    \refstepcounter{blind@pangramcount}%
  \ifnum\blind@countxx<\value{blind@pangrammax}\advance\blind@countxx by 1 %
  \repeat%
  \setcounter{blind@pangrammax}{\value{blindtext}}%
  }% \blindtext@text  
  \def\blindtext@parstart{}%no change for pangram option
}
\fi %option pangram
%    \end{macrocode}
%
% ^^A %%%%%%%%%% End English Texts %%%%%%%%%%%%%%%%
%

% %
%
% ^^A This document is generated by mk_blindtext_texts.rb
%
%
% \subsection{German Texts (babel: german)}
% \changes{V1.9e}{2011-12-09}{Correction German}
% Thanks to Felix Lehmann for corrections.
%
%    \begin{macro}{\blindtext@german}
%    Define flag, so we can check if language is defined.
%    \begin{macrocode}
\def\blindtext@german{}
%    \end{macrocode}
%    \end{macro}
%
% Define the default blind text for German.
%    \begin{macrocode}
\blind@addtext{german}{%
  \def\blindtext@text{%
    Dies hier ist ein Blindtext zum Testen von
    Textausgaben\blindtext@endsentence Wer diesen Text liest, ist selbst
    schuld\blindtext@endsentence Der Text gibt lediglich den Grauwert
    der Schrift an\blindtext@endsentence Ist das wirklich so? Ist es
    gleich\-g\"ul\-tig, ob ich schreibe: \glqq Dies ist ein
    Blindtext\grqq\ oder \glqq Huardest gefburn\grqq ? Kjift --
    mitnichten! Ein Blindtext bietet mir wichtige
    Informationen\blindtext@endsentence An ihm messe ich die
    \blindmarkup{Lesbarkeit einer Schrift}, ihre Anmutung, wie
    harmonisch die Figuren zueinander stehen und pr\"u\-fe, wie breit
    oder schmal sie l\"auft\blindtext@endsentence Ein Blindtext sollte
    m\"og\-lichst \blindmarkup{viele verschiedene Buchstaben} enthalten
    und in der Originalsprache gesetzt sein\blindtext@endsentence Er
    mu\ss\ keinen Sinn ergeben, sollte aber lesbar
    sein\blindtext@endsentence Fremdsprachige Texte wie \glqq Lorem
    ipsum\grqq\ dienen nicht dem eigentlichen Zweck, da sie eine falsche
    Anmutung vermitteln\blindtext@endsentence%
  }% \blindtext@text
}
%    \end{macrocode}
%
%
% \changes{V1.9e}{2011-12-09}{Default paragraph start for German}
% Define different paragraph starts for second and later paragraphs.
% The first paragraph gets no special start.
%    \begin{macrocode}
\blind@addtext{german}{%
  \def\blindtext@parstart{%
      \ifcase\value{blind@countparstart}\or
Das hier ist der zweite Absatz.\or
Und nun folgt -- ob man es glaubt oder nicht --  der dritte Absatz.\or
Nach diesem vierten Absatz beginnen wir eine neue Z\"ahlung.\or
        \setcounter{blind@countparstart}{0}
      \fi
      \stepcounter{blind@countparstart}
  }% \blindtext@parstart
}
%    \end{macrocode}
%
% Define counters for list environments.
%    \begin{macrocode}
\blind@addtext{german}{%
  \def\blindtext@count{%
    \ifcase\value{blind@listcount}\or
      Erster\or Zweiter\or Dritter\or Vierter\or F{\"u}nfter\or
      Sechster\or Siebter\or Achter\or Neunter\or Zehnter\or
      Elfter\or Zw{\"o}lfter\or Dreizehnter\or Vierzehnter%
    \else
      Noch ein%
    \fi
  }% \blindtext@count
  \def\blindtext@item{Listenpunkt, Stufe~\arabic{blind@levelcount}}%
}%\addto\extrasgerman
%    \end{macrocode}
%
% Define title lines for German.
%    \begin{macrocode}
\blind@addtext{german}{%
  \def\blindtext@heading{{\"U}berschrift auf Ebene
\xspace}%
  \def\blindtext@list{Listen}%
  \def\blindtext@listEx{Beispiel einer Liste\xspace}%
}%\addto\extrasgerman
%    \end{macrocode}
%
% Add the title for |\blindmathpaper|.
%    \begin{macrocode}
\blind@addtext{german}{%
    \def\blindtext@blindmath{Blindtext mit mathematischen Formeln}%
}%\addto\extrasgerman
%    \end{macrocode}
%
%
% Define the bible-option text for german.
%    \begin{macrocode}
\ifblindbible
\blind@addtext{german}{%
  \def\blindtext@text{%
    Da sprach Gott der Herr zu der Schlange: Weil du solches getan hast,
    seist du verflucht vor allem Vieh und vor allen Tieren auf dem
    Felde. Auf deinem Bauche sollst du gehen und Erde essen dein Leben
    lang.
    Gott sprach zu Mose: \glqq Ich werde sein, der Ich sein werde.\grqq\
    Und sprach: Also sollst du den Kindern Israel sagen: \glqq Ich werde
    sein\grqq\ hat mich zu euch gesandt\ldots
    und er soll davon opfern ein Opfer dem Herrn, n\"amlich das Fett,
    welches die Eingeweide bedeckt, und alles Fett am Eingeweide,\ldots
    Und der HERR redete mit Mose in der W\"uste Sinai und sprach:
    Jair, der Sohn Manasses, nahm die ganze Gegend Argob bis an die
    Grenze der Gessuriter und Maachathiter und hiess das Basan nach
    seinem Namen D\"orfer Jairs bis auf den heutigen Tag.%
  }% \blindtext@text
  \def\blindtext@parstart{}%no change for bible option
}
\fi %\ifbible
%    \end{macrocode}
%
% Define the random-option text for german.
%    \begin{macrocode}
\ifblindrandom
  \blind@addtext{german}{%
      \setcounter{blindtext}{17}
      \def\blindtext@text{%
      \blind@countxx=1 %
      \loop  
        \ifcase\value{blind@randomcount}%
Dies hier ist ein Blindtext zum Testen von
Textausgaben\blindtext@endsentence
\or Gerne werden Pangramme als Blindtexte
verwendet\blindtext@endsentence
\or Das griechische Wort Pangramm (oder holoalphabetischer Satz)
bezeichnet einen Satz, der alle Buchstaben des Alphabets
enth\"alt\blindtext@endsentence
\or Wobei man \glqq alle Buchstaben\grqq\ mit und ohne Umlaute z\"ahlen
kann\blindtext@endsentence
\or Aber das soll uns hier nicht k\"ummern, eigentlich wollen wir doch
eine Geschichte erz\"ahlen\blindtext@endsentence
\or Aber wozu wollen wir eine Geschichte erz\"ahlen?\xspace
\or Ach ja, wir brauchen Text um das Layout dieses Textes zu p\"ufen --
dazu nimmt man meist einen Blindtext\blindtext@endsentence%
      \setcounter{blind@randomcount}{-1}%
      \fi%
      \refstepcounter{blind@randomcount}%
    \ifnum\blind@countxx<\value{blind@randommax}\advance\blind@countxx by 1 %
    \repeat%
    \setcounter{blind@randommax}{\value{blindtext}}%
    }% \blindtext@text  
    \def\blindtext@parstart{}%no change for random option
  }
\fi %option random
%    \end{macrocode}
%
% Define the pangram-option text for german.
%    \begin{macrocode}
\ifblindpangram
\blind@addtext{german}{%
    \setcounter{blindtext}{5}
    \def\blindtext@text{%
    \blind@countxx=1 %
    \loop  
      \ifcase\value{blind@pangramcount}%
Franz jagt im komplett verwahrlosten Taxi quer durch
Bayern\blindtext@endsentence
\or Zw\"olf Boxk\"ampfer jagen Viktor quer \"uber den gro{\ss}en Sylter
Deich\blindtext@endsentence
\or Vogel Quax zwickt Johnys Pferd Bim\blindtext@endsentence
\or Sylvia wagt quick den Jux bei Pforzheim\blindtext@endsentence
\or Prall vom Whisky flog Quax den Jet zu Bruch\blindtext@endsentence
\or Jeder wackere Bayer vertilgt bequem zwo Pfund
Kalbshaxen\blindtext@endsentence
\or Stanleys Expeditionszug quer durch Afrika wird von jedermann
bewundert\blindtext@endsentence%
    \setcounter{blind@pangramcount}{-1}%
    \fi%
    \refstepcounter{blind@pangramcount}%
  \ifnum\blind@countxx<\value{blind@pangrammax}\advance\blind@countxx by 1 %
  \repeat%
  \setcounter{blind@pangrammax}{\value{blindtext}}%
  }% \blindtext@text  
  \def\blindtext@parstart{}%no change for pangram option
}
\fi %option pangram
%    \end{macrocode}
% If the package \Lpack{german} is loaded, select the language.
%    \begin{macrocode}
\@ifpackageloaded{german}{\selectlanguage{german}}{}
%    \end{macrocode}
%
% ^^A %%%%%%%%%% End German Texts %%%%%%%%%%%%%%%%
%

% %
%
% ^^A This document is generated by mk_blindtext_texts.rb
%
%
% \subsection{German -- New Orthography (babel: ngerman)}
% \changes{V1.9e}{2011-12-09}{Correction NGerman}
% Thanks to Felix Lehmann for corrections.
%
%    \begin{macro}{\blindtext@ngerman}
%    Define flag, so we can check if language is defined.
%    \begin{macrocode}
\def\blindtext@ngerman{}
%    \end{macrocode}
%    \end{macro}
%
% Define the default blind text for Ngerman.
%    \begin{macrocode}
\blind@addtext{ngerman}{%
  \def\blindtext@text{%
    Dies hier ist ein Blindtext zum Testen von
    Textausgaben\blindtext@endsentence Wer diesen Text liest, ist selbst
    schuld\blindtext@endsentence Der Text gibt lediglich den Grauwert
    der Schrift an\blindtext@endsentence Ist das wirklich so? Ist es
    gleich\-g\"ul\-tig, ob ich schreibe: \glqq Dies ist ein
    Blindtext\grqq\ oder \glqq Huardest gefburn\grqq ? Kjift --
    mitnichten! Ein Blindtext bietet mir wichtige
    Informationen\blindtext@endsentence An ihm messe ich die
    \blindmarkup{Lesbarkeit einer Schrift}, ihre Anmutung, wie
    harmonisch die Figuren zueinander stehen und pr\"u\-fe, wie breit
    oder schmal sie l\"auft\blindtext@endsentence Ein Blindtext sollte
    m\"og\-lichst viele \blindmarkup{verschiedene Buchstaben} enthalten
    und in der Originalsprache gesetzt sein\blindtext@endsentence Er
    muss keinen Sinn ergeben, sollte aber lesbar
    sein\blindtext@endsentence Fremdsprachige Texte wie \glqq Lorem
    ipsum\grqq\ dienen nicht dem eigentlichen Zweck, da sie eine falsche
    Anmutung vermitteln\blindtext@endsentence%
  }% \blindtext@text
}
%    \end{macrocode}
%
%
% \changes{V1.9e}{2011-12-09}{Default paragraph start for Ngerman}
% Define different paragraph starts for second and later paragraphs.
% The first paragraph gets no special start.
%    \begin{macrocode}
\blind@addtext{ngerman}{%
  \def\blindtext@parstart{%
      \ifcase\value{blind@countparstart}\or
Das hier ist der zweite Absatz.\or
Und nun folgt -- ob man es glaubt oder nicht --  der dritte Absatz.\or
Nach diesem vierten Absatz beginnen wir eine neue Z\"ahlung.\or
        \setcounter{blind@countparstart}{0}
      \fi
      \stepcounter{blind@countparstart}
  }% \blindtext@parstart
}
%    \end{macrocode}
%
% Define counters for list environments.
%    \begin{macrocode}
\blind@addtext{ngerman}{%
  \def\blindtext@count{%
    \ifcase\value{blind@listcount}\or
      Erster\or Zweiter\or Dritter\or Vierter\or F{\"u}nfter\or
      Sechster\or Siebter\or Achter\or Neunter\or Zehnter\or
      Elfter\or Zw{\"o}lfter\or Dreizehnter\or Vierzehnter%
    \else
      Noch ein%
    \fi
  }% \blindtext@count
  \def\blindtext@item{Listenpunkt, Stufe~\arabic{blind@levelcount}}%
}%\addto\extrasngerman
%    \end{macrocode}
%
% Define title lines for Ngerman.
%    \begin{macrocode}
\blind@addtext{ngerman}{%
  \def\blindtext@heading{{\"U}berschrift auf Ebene\xspace}%
  \def\blindtext@list{Listen}%
  \def\blindtext@listEx{Beispiel einer Liste\xspace}%
}%\addto\extrasngerman
%    \end{macrocode}
%
% Add the title for |\blindmathpaper|.
%    \begin{macrocode}
\blind@addtext{ngerman}{%
    \def\blindtext@blindmath{Blindtext mit mathematischen Formeln}%
}%\addto\extrasngerman
%    \end{macrocode}
%
%
% Define the bible-option text for ngerman.
%    \begin{macrocode}
\ifblindbible
\blind@addtext{ngerman}{%
  \def\blindtext@text{%
    Da sprach Gott der Herr zu der Schlange: Weil du solches getan hast,
    seist du verflucht vor allem Vieh und vor allen Tieren auf dem
    Felde. Auf deinem Bauche sollst du gehen und Erde essen dein Leben
    lang.
    Gott sprach zu Mose: \glqq Ich werde sein, der Ich sein werde.\grqq\
    Und sprach: Also sollst du den Kindern Israel sagen: \glqq Ich werde
    sein\grqq\ hat mich zu euch gesandt\ldots
    und er soll davon opfern ein Opfer dem Herrn, n\"amlich das Fett,
    welches die Eingeweide bedeckt, und alles Fett am Eingeweide,\ldots
    Und der HERR redete mit Mose in der W\"uste Sinai und sprach:
    Jair, der Sohn Manasses, nahm die ganze Gegend Argob bis an die
    Grenze der Gessuriter und Maachathiter und hiess das Basan nach
    seinem Namen D\"orfer Jairs bis auf den heutigen Tag.%
  }% \blindtext@text
  \def\blindtext@parstart{}%no change for bible option
}
\fi %\ifbible
%    \end{macrocode}
%
% Define the random-option text for ngerman.
%    \begin{macrocode}
\ifblindrandom
  \blind@addtext{ngerman}{%
      \setcounter{blindtext}{17}
      \def\blindtext@text{%
      \blind@countxx=1 %
      \loop  
        \ifcase\value{blind@randomcount}%
Dies hier ist ein Blindtext zum Testen von
Textausgaben\blindtext@endsentence
\or Gerne werden Pangramme als Blindtexte
verwendet\blindtext@endsentence
\or Das griechische Wort Pangramm (oder holoalphabetischer Satz)
bezeichnet einen Satz, der alle Buchstaben des Alphabets
enth\"alt\blindtext@endsentence
\or Wobei man \glqq alle Buchstaben\grqq\ mit und ohne Umlaute z\"ahlen
kann\blindtext@endsentence
\or Aber das soll uns hier nicht k\"ummern, eigentlich wollen wir doch
eine Geschichte erz\"ahlen\blindtext@endsentence
\or Aber wozu wollen wir eine Geschichte erz\"ahlen?\xspace
\or Ach ja, wir brauchen Text um das Layout dieses Textes zu p\"ufen --
dazu nimmt man meist einen Blindtext\blindtext@endsentence%
      \setcounter{blind@randomcount}{-1}%
      \fi%
      \refstepcounter{blind@randomcount}%
    \ifnum\blind@countxx<\value{blind@randommax}\advance\blind@countxx by 1 %
    \repeat%
    \setcounter{blind@randommax}{\value{blindtext}}%
    }% \blindtext@text  
    \def\blindtext@parstart{}%no change for random option
  }
\fi %option random
%    \end{macrocode}
%
% Define the pangram-option text for ngerman.
%    \begin{macrocode}
\ifblindpangram
\blind@addtext{ngerman}{%
    \setcounter{blindtext}{5}
    \def\blindtext@text{%
    \blind@countxx=1 %
    \loop  
      \ifcase\value{blind@pangramcount}%
Franz jagt im komplett verwahrlosten Taxi quer durch
Bayern\blindtext@endsentence
\or Zw\"olf Boxk\"ampfer jagen Viktor quer \"uber den gro{\ss}en Sylter
Deich\blindtext@endsentence
\or Vogel Quax zwickt Johnys Pferd Bim\blindtext@endsentence
\or Sylvia wagt quick den Jux bei Pforzheim\blindtext@endsentence
\or Prall vom Whisky flog Quax den Jet zu Bruch\blindtext@endsentence
\or Jeder wackere Bayer vertilgt bequem zwo Pfund
Kalbshaxen\blindtext@endsentence
\or Stanleys Expeditionszug quer durch Afrika wird von jedermann
bewundert\blindtext@endsentence%
    \setcounter{blind@pangramcount}{-1}%
    \fi%
    \refstepcounter{blind@pangramcount}%
  \ifnum\blind@countxx<\value{blind@pangrammax}\advance\blind@countxx by 1 %
  \repeat%
  \setcounter{blind@pangrammax}{\value{blindtext}}%
  }% \blindtext@text  
  \def\blindtext@parstart{}%no change for pangram option
}
\fi %option pangram
%    \end{macrocode}
%
% If the package \Lpack{ngerman} is loaded, select the language.
%    \begin{macrocode}
\@ifpackageloaded{ngerman}{\selectlanguage{ngerman}}{}
%    \end{macrocode}
%
% ^^A %%%%%%%%%% End German -- New Orthography %%%%%%%%%%%%%%%%
%

% %
%
% ^^A This document is generated by mk_blindtext_texts.rb
%
%
% \subsection{American Texts (English) (babel: american)}
% \changes{V1.9e}{2011-12-09}{Add American}
% American is a copy of English. This definition is added to allow the usage of Amrican with ba
% Thanks to Karl Voit for the hint.
%
%    \begin{macro}{\blindtext@american}
%    Define flag, so we can check if language is defined.
%    \begin{macrocode}
\def\blindtext@american{}
%    \end{macrocode}
%    \end{macro}
%
% Define the default blind text for American.
%    \begin{macrocode}
\blind@addtext{american}{%
  \def\blindtext@text{%
    Hello, here is some text without a meaning\blindtext@endsentence
    This text should show what a printed text will look like at this
    place\blindtext@endsentence If you read this text, you will get no
    information\blindtext@endsentence Really? Is there no information?
    Is there a difference between this text and some nonsense like
    ``Huardest gefburn''? Kjift -- not at all! A blind text
    \blindmarkup{like this} gives you information  about the selected
    font, how the letters are written and an impression  of the
    look\blindtext@endsentence This text should contain \blindmarkup{all
    letters of the alphabet} and it should be written in of the original
    language\blindtext@endsentence There is no need for special
    contents, but the length of words should match the
    language\blindtext@endsentence%
  }% \blindtext@text
}
%    \end{macrocode}
%
%
% \changes{V1.9e}{2011-12-09}{Default paragraph start for American}
% Define different paragraph starts for second and later paragraphs.
% The first paragraph gets no special start.
%    \begin{macrocode}
\blind@addtext{american}{%
  \def\blindtext@parstart{%
      \ifcase\value{blind@countparstart}\or
This is the second paragraph.\or
And after the second paragraph follows the third paragraph.\or
After this fourth paragraph, we start a new paragraph sequence.\or
        \setcounter{blind@countparstart}{0}
      \fi
      \stepcounter{blind@countparstart}
  }% \blindtext@parstart
}
%    \end{macrocode}
%
% Define counters for list environments.
%    \begin{macrocode}
\blind@addtext{american}{%
  \def\blindtext@count{%
    \ifcase\value{blind@listcount}\or
      First\or Second\or Third\or Fourth\or Fifth\or
      Sixth\or Seventh\or Eighth\or Ninth\or Tenth\or
      Eleventh\or Twelfth%
    \else
      Another%
    \fi
  }% \blindtext@count
  \def\blindtext@item{item in a list}%
}%\addto\extrasamerican
%    \end{macrocode}
%
% Define title lines for American.
%    \begin{macrocode}
\blind@addtext{american}{%
  \def\blindtext@heading{Heading on level\xspace}%
  \def\blindtext@list{Lists}%
  \def\blindtext@listEx{Example for list\xspace}%
}%\addto\extrasamerican
%    \end{macrocode}
%
% Add the title for |\blindmathpaper|.
%    \begin{macrocode}
\blind@addtext{american}{%
    \def\blindtext@blindmath{Some blind text with math formulas}%
}%\addto\extrasamerican
%    \end{macrocode}
%
%
% Define the bible-option text for american.
%    \begin{macrocode}
\ifblindbible
\blind@addtext{american}{%
  \def\blindtext@text{%
    And the Lord God said unto the serpent, Because thou hast done this,
    thou art cursed above all cattle, and above every beast of the
    field; upon thy belly shalt thou go, and dust shalt thou eat all the
    days of thy life:
    And God said unto Moses, `I am that I am': and he said, Thus shalt
    thou say unto the children of Israel, `I am' hath sent me unto you.
    And he shall offer thereof his offering, even an offering made by
    fire unto the Lord; the fat that covereth the inwards, and all the
    fat that is upon the inwards,\ldots
    And the Lord spake unto Moses in the wilderness of Sinai,
    saying,\ldots
    Jair the son of Manasseh took all the country of Argob unto the
    coasts of Geshuri and Maachathi; and called them after his own name,
    Bashanhavothjair, unto this day.%
  }% \blindtext@text
  \def\blindtext@parstart{}%no change for bible option
}
\fi %\ifbible
%    \end{macrocode}
%
% Define the random-option text for american.
%    \begin{macrocode}
\ifblindrandom
  \PackageWarning{blindtext}{Option random not defined for american\MessageBreak}%
  \blind@addtext{american}{%
    \setcounter{blindtext}{1}
  }
\fi %option random
%    \end{macrocode}
%
% Define the pangram-option text for american.
%    \begin{macrocode}
\ifblindpangram
\blind@addtext{american}{%
    \setcounter{blindtext}{5}
    \def\blindtext@text{%
    \blind@countxx=1 %
    \loop  
      \ifcase\value{blind@pangramcount}%
The quick brown fox jumps over the lazy dog\blindtext@endsentence
\or Jackdaws love my big Sphinx of Quartz\blindtext@endsentence
\or Pack my box with five dozen liquor jugs\blindtext@endsentence
\or The five boxing wizards jump quickly\blindtext@endsentence
\or Sympathizing would fix Quaker objectives\blindtext@endsentence
\or Many-wived Jack laughs at probes of sex quiz\blindtext@endsentence
\or Turgid saxophones blew over Mick's jazzy quaff\blindtext@endsentence
\or Playing jazz vibe chords quickly excites my
wife\blindtext@endsentence
\or A large fawn jumped quickly over white zinc
boxes\blindtext@endsentence
\or Exquisite farm wench gives body jolt to prize
stinker\blindtext@endsentence
\or Jack amazed a few girls by dropping the antique onyx vase!\xspace%
    \setcounter{blind@pangramcount}{-1}%
    \fi%
    \refstepcounter{blind@pangramcount}%
  \ifnum\blind@countxx<\value{blind@pangrammax}\advance\blind@countxx by 1 %
  \repeat%
  \setcounter{blind@pangrammax}{\value{blindtext}}%
  }% \blindtext@text  
  \def\blindtext@parstart{}%no change for pangram option
}
\fi %option pangram
%    \end{macrocode}
%
% ^^A %%%%%%%%%% End American Texts (English) %%%%%%%%%%%%%%%%
%

% %
%
% ^^A This document is generated by mk_blindtext_texts.rb
%
%
% \subsection{Catalan Texts (babel: catalan)}
% \changes{V1.9e}{2011-12-09}{Add Catalan}
% Thanks to Joan Queralt.
%
%    \begin{macro}{\blindtext@catalan}
%    Define flag, so we can check if language is defined.
%    \begin{macrocode}
\def\blindtext@catalan{}
%    \end{macrocode}
%    \end{macro}
%
% Define the default blind text for Catalan.
%    \begin{macrocode}
\blind@addtext{catalan}{%
  \def\blindtext@text{%
    Qu\`{e} \'es aix\`{o}?\blindtext@endsentence \'Es la meva primera
    frase des de fa anys: Lorem ipsum dolor sit amet, consectetuer
    adipiscing elit\blindtext@endsentence Etiam lobortis facilisis
    sem\blindtext@endsentence Nullam nec mi et neque pharetra
    sollicitudin\blindtext@endsentence Praesent imperdiet mi nec
    ante\blindtext@endsentence  Donec ullamcorper, \blindmarkup{felis
    non sodales commodo}, lectus velit ultrices augue, a dignissim nibh
    lectus placerat pede\blindtext@endsentence Vivamus nunc nunc,
    molestie ut, ultricies vel, \blindmarkup{semper in},
    velit\blindtext@endsentence  Ut porttitor\blindtext@endsentence
    Praesent in sapien\blindtext@endsentence%
  }% \blindtext@text
}
%    \end{macrocode}
%
%
% \changes{V1.9e}{2011-12-09}{Default paragraph start for Catalan}
% Define different paragraph starts for second and later paragraphs.
% The first paragraph gets no special start.
%    \begin{macrocode}
\blind@addtext{catalan}{%
  \def\blindtext@parstart{%
      \ifcase\value{blind@countparstart}\or
Aquest \'es el segon par\`agraf\blindtext@endsentence\or
I despr\'es del segon ve el tercer par\`agraf\blindtext@endsentence\or
Despr\'es del quart par\`agraf comencem una nova tanda de nous par\`agrafs\blindtext@endsentence\or
        \setcounter{blind@countparstart}{0}
      \fi
      \stepcounter{blind@countparstart}
  }% \blindtext@parstart
}
%    \end{macrocode}
%
% Define counters for list environments.
%    \begin{macrocode}
\blind@addtext{catalan}{%
  \def\blindtext@count{%
    \ifcase\value{blind@listcount}\or
      Primer\or Segon\or Tercer\or Quart\or Cinqu\`e\or
      Sis\`e\or Set\`e\or Vuit\`e\or Nov\`e\or Des\`e\or
      Onz\`e\or Dotuz\`e%
    \else
      Altres%
    \fi
  }% \blindtext@count
  \def\blindtext@item{punt d'una llista}%
}%\addto\extrascatalan
%    \end{macrocode}
%
% Define title lines for Catalan.
%    \begin{macrocode}
\blind@addtext{catalan}{%
  \def\blindtext@heading{T\'itol de nivell\xspace}%
  \def\blindtext@list{Llistes}%
  \def\blindtext@listEx{Exemple de llista\xspace}%
}%\addto\extrascatalan
%    \end{macrocode}
%
% Add the title for |\blindmathpaper|.
%    \begin{macrocode}
\blind@addtext{catalan}{%
    \def\blindtext@blindmath{Alguns textos amb f\'ormules matem\`atiques.}%
}%\addto\extrascatalan
%    \end{macrocode}
%
%
% Define the bible-option text for catalan.
%    \begin{macrocode}
\ifblindbible
\blind@addtext{catalan}{%
  \def\blindtext@text{%
    Jahv\`e D\'eu digu\'e a la serp16 : \flqq Perqu\`e has fet aix\`o,
    ser\`as male\"{i}da entre totes les b\`esties i tots els animals
    salvatges. T'arrossegar\`as damunt del ventre i menjar\`as pols tot
    el temps de la teva vida.
    Llavors D\'eu digu\'e a Mois\`es: \flqq Jo s\'ec, el qui s\'ec.\frqq
    I afeg\'i: \flqq Aix\'i parlar\`as als israelites: Jo s\'ec m'ha
    enviat a vosaltres\frqq
    Com a combusti\'e per a Jahv\`e, n'oferir\`a el greix que cobreix
    les entranyes i tot el greix de damunt les entranyes;\ldots
    Jahv\`e va dir a Mois\`es, a la muntanya del Sina\'i: 1\ldots
    Ja\"{i}r, fill de Manas\'es, s'apoder\`a de tota la regi\'e d'Argob
    fins a la frontera dels guesurites i dels macatites, i don\`a a
    Basan el seu nom d'Havot-Ja\"{i}r, que ha quedat fins avu\'i.%
  }% \blindtext@text
  \def\blindtext@parstart{}%no change for bible option
}
\fi %\ifbible
%    \end{macrocode}
%
% Define the random-option text for catalan.
%    \begin{macrocode}
\ifblindrandom
  \PackageWarning{blindtext}{Option random not defined for catalan\MessageBreak}%
  \blind@addtext{catalan}{%
    \setcounter{blindtext}{1}
  }
\fi %option random
%    \end{macrocode}
%
% Define the pangram-option text for catalan.
%    \begin{macrocode}
\ifblindpangram
\blind@addtext{catalan}{%
    \setcounter{blindtext}{5}
    \def\blindtext@text{%
    \blind@countxx=1 %
    \loop  
      \ifcase\value{blind@pangramcount}%
Jove xef, porti whisky amb quinze gla\c{c}ons d'hidrogen, coi!
\or Aqueix betzol, Jan, comprava whisky de figa\blindtext@endsentence
\or Zel de grum: quetxup, whisky, caf\`e, bon vi; ja!
\or Coi! quinze jans golafres de X\`ativa, beuen whisky a
pams\blindtext@endsentence%
    \setcounter{blind@pangramcount}{-1}%
    \fi%
    \refstepcounter{blind@pangramcount}%
  \ifnum\blind@countxx<\value{blind@pangrammax}\advance\blind@countxx by 1 %
  \repeat%
  \setcounter{blind@pangrammax}{\value{blindtext}}%
  }% \blindtext@text  
  \def\blindtext@parstart{}%no change for pangram option
}
\fi %option pangram
%    \end{macrocode}
%
% ^^A %%%%%%%%%% End Catalan Texts %%%%%%%%%%%%%%%%
%

% %
%
% ^^A This document is generated by mk_blindtext_texts.rb
%
%
% \subsection{Latin Texts (babel: latin)}
% 
% I don't speak Latin, but I think the classic "Lorem ipsum" should be available.
% There is no inline math supported for this "Lorem ipsum".
% 
% The following "Latin" texts are not really correct Latin.
% If you want correct texts, please provide them to me.
% 
% \changes{V1.9e}{2011-12-31}{Asterix citations}
% Latin with option \emph{random} use some citations from Asterix.
%
%    \begin{macro}{\blindtext@latin}
%    Define flag, so we can check if language is defined.
%    \begin{macrocode}
\def\blindtext@latin{}
%    \end{macrocode}
%    \end{macro}
%
% Define the default blind text for Latin.
%    \begin{macrocode}
\blind@addtext{latin}{%
  \def\blindtext@text{%
    Lorem ipsum dolor sit amet, consectetuer adipiscing elit. Etiam
    lobortis facilisis sem. Nullam nec mi et neque pharetra
    sollicitudin. Praesent imperdiet mi nec ante. Donec ullamcorper,
    felis non sodales commodo, lectus velit ultrices augue, a dignissim
    nibh lectus placerat pede. Vivamus nunc nunc, molestie ut, ultricies
    vel, semper in, velit. Ut porttitor. Praesent in sapien.
    \blindmarkup{Lorem ipsum} dolor sit amet, consectetuer adipiscing
    elit. Duis fringilla tristique neque. Sed interdum libero ut metus.
    Pellentesque placerat. Nam rutrum augue a leo. Morbi sed elit sit
    amet ante lobortis sollicitudin. Praesent blandit blandit mauris.
    Praesent lectus tellus, \blindmarkup{aliquet aliquam}, luctus a,
    egestas a, turpis. Mauris lacinia lorem sit amet ipsum. Nunc quis
    urna dictum turpis accumsan semper.%
  }% \blindtext@text
}
%    \end{macrocode}
%
% Define counters for list environments.
%    \begin{macrocode}
\blind@addtext{latin}{%
  \def\blindtext@count{%
    \ifcase\value{blind@listcount}\or
      Primus\or Duo\or Tres\or Quattuor\or Quinque\or
      Sex\or Septem\or Octo\or Novem\or Decem\\or
      Undecim\or Duodecim%
    \else
      Nova%
    \fi
  }% \blindtext@count
  \def\blindtext@item{, altum~\arabic{blind@levelcount}}%
}%\addto\extraslatin
%    \end{macrocode}
%
% Define title lines for Latin.
%    \begin{macrocode}
\blind@addtext{latin}{%
  \def\blindtext@heading{Sectio\xspace}%
  \def\blindtext@list{Caudex}%
  \def\blindtext@listEx{Exemplum caudex\xspace}%
}%\addto\extraslatin
%    \end{macrocode}
%
% Add the title for |\blindmathpaper|.
%    \begin{macrocode}
\blind@addtext{latin}{%
    \def\blindtext@blindmath{Mathematica}%
}%\addto\extraslatin
%    \end{macrocode}
%
%
% Define the bible-option text for latin.
%    \begin{macrocode}
\ifblindbible
\blind@addtext{latin}{%
  \def\blindtext@text{%
    et ait Dominus Deus ad serpentem quia fecisti hoc maledictus es
    inter omnia animantia et bestias terrae super pectus tuum gradieris
    et terram comedes cunctis diebus vitae tuae
    dixit Deus ad Mosen ego sum qui sum ait sic dices filiis Israhel qui
    est misit me ad vos
    tollentque ex ea in pastum ignis dominici adipem qui operit ventrem
    et qui tegit universa vitalia
    Iocutus est Dominus ad Mosen in deserto Sinai dicens
    Iair filius Manasse possedit omnem regionem Argob usque ad terminos
    Gesuri et Machathi vocavitque ex nomine suo Basan Avothiair id est
    villas Iair usque in praesentem diem%
  }% \blindtext@text
  \def\blindtext@parstart{}%no change for bible option
}
\fi %\ifbible
%    \end{macrocode}
%
% Define the random-option text for latin.
%    \begin{macrocode}
\ifblindrandom
  \blind@addtext{latin}{%
      \setcounter{blindtext}{17}
      \def\blindtext@text{%
      \blind@countxx=1 %
      \loop  
        \ifcase\value{blind@randomcount}%
Ab imo pectore\blindtext@endsentence
\or Acta est fabula\blindtext@endsentence
\or Ad augusta per angusta!\xspace
\or Ad gladios!\xspace
\or Alea iacta est\blindtext@endsentence
\or Argumentum baculinum!\xspace
\or Audaces fortuna juvat!\xspace
\or Auri sacra fames!\xspace
\or Aut Caesar, aut nihil!\xspace
\or Ave C\"asar, lucrifacturi te salutant!\xspace
\or Beati Asterixem possidentes!\xspace
\or Beati pauperes spiritu\blindtext@endsentence
\or Bis repetita non placent\blindtext@endsentence
\or Carpe diem\blindtext@endsentence
\or Cautela abundans non nocet\blindtext@endsentence
\or Cogito, ergo sum\blindtext@endsentence
\or Concursu!\xspace
\or Condicio sine qua non\blindtext@endsentence
\or Contraria contrariis curantur!\xspace
\or Similia similibus curantur\blindtext@endsentence
\or Da capo!\xspace
\or Ceterum censeo Carthaginem esse delendam\blindtext@endsentence
\or Desinit in piscem mulier formosa superne!\xspace
\or Diem perdidi!\xspace
\or Dignus est intrare\blindtext@endsentence
\or Donec eris felix, multos numerabis amicos\blindtext@endsentence
\or Tempora si fuerint nubila, solus eris\blindtext@endsentence
\or Dulce et decorum est pro patria mori\blindtext@endsentence
\or Errare humanum est\blindtext@endsentence
\or Et nunc reges, intelligite erudimini qui judicatis
terram\blindtext@endsentence
\or Exegi monumentum aere perennius\blindtext@endsentence
\or Felix, qui potuit rerum cognoscere\blindtext@endsentence
\or Felix qui potuit rerum cognocscere causas!\xspace
\or Fluctuat nec mergitur!\xspace
\or Ipso facto!\xspace
\or Ira furor brevis est\blindtext@endsentence
\or Ita deis placuit!\xspace
\or Ita est!\xspace
\or Legio expedita!\xspace
\or Leontes te devorant \blindtext@endsentence
\or Major e longinquo reverentia\blindtext@endsentence
\or Mens sana in corpore sano\blindtext@endsentence
\or Morituri te salutant!\xspace
\or Nihil conveniens decretis ejus!\xspace
\or Non licet omnibus adire Brivatum\blindtext@endsentence
\or Non licet omnibus adire Corinthum\blindtext@endsentence
\or Non omnia possumus omnes!\xspace
\or O fortunatos nimium, sua si bona norint, agricolas!\xspace
\or Qui habet aures audiendi, audiat!\xspace
\or Quod erat demonstrandum\blindtext@endsentence
\or Quot capita tot census!\xspace
\or Sic transit gloria mundi\blindtext@endsentence
\or Summum jus, summa injuria!\xspace
\or Ubi solitudinem faciunt, pacem appellant\blindtext@endsentence
\or Vanitas vanitatum et omnia vanitas\blindtext@endsentence
\or Victrix causa diis placuit, sed victa Catoni\blindtext@endsentence
\or Video meliora proboque deteriora sequor\blindtext@endsentence
\or Vinum et musica laetificant cor\blindtext@endsentence%
      \setcounter{blind@randomcount}{-1}%
      \fi%
      \refstepcounter{blind@randomcount}%
    \ifnum\blind@countxx<\value{blind@randommax}\advance\blind@countxx by 1 %
    \repeat%
    \setcounter{blind@randommax}{\value{blindtext}}%
    }% \blindtext@text  
    \def\blindtext@parstart{}%no change for random option
  }
\fi %option random
%    \end{macrocode}
%
% Define the pangram-option text for latin.
%    \begin{macrocode}
\ifblindpangram
\blind@addtext{latin}{%
    \setcounter{blindtext}{5}
    \def\blindtext@text{%
    \blind@countxx=1 %
    \loop  
      \ifcase\value{blind@pangramcount}%
Sic fugiens, dux, zelotypos quam karus haberis\blindtext@endsentence
\or Duc zephire exurgens currum cum flatibus
\ae{}quor\blindtext@endsentence
\or Vix phlegeton zephiri qu\ae{}rens modo flabra
mycillo\blindtext@endsentence%
    \setcounter{blind@pangramcount}{-1}%
    \fi%
    \refstepcounter{blind@pangramcount}%
  \ifnum\blind@countxx<\value{blind@pangrammax}\advance\blind@countxx by 1 %
  \repeat%
  \setcounter{blind@pangrammax}{\value{blindtext}}%
  }% \blindtext@text  
  \def\blindtext@parstart{}%no change for pangram option
}
\fi %option pangram
%    \end{macrocode}
%
% ^^A %%%%%%%%%% End Latin Texts %%%%%%%%%%%%%%%%
%

% %
%
% ^^A This document is generated by mk_blindtext_texts.rb
%
%
% \subsection{FrenchTexts (babel: french)}
% \changes{V1.9b}{2009-06-05}{Support French - interim version}
% \changes{V1.9b}{2009-12-29}{Correction French}
% This text is just an interim solution until I get a correct text.
% To fill up the text the Latin Lorem lipsum is used.
% \changes{V1.9e}{2011-12-11}{Add Lorem Lipsum to French}
%
%    \begin{macro}{\blindtext@french}
%    Define flag, so we can check if language is defined.
%    \begin{macrocode}
\def\blindtext@french{}
%    \end{macrocode}
%    \end{macro}
%
% Define the default blind text for French.
%    \begin{macrocode}
\blind@addtext{french}{%
  \def\blindtext@text{%
    Qu'est que c'est?\blindtext@endsentence  C'est une phrase
    fran\c{c}ais \blindmarkup{avant le lorem
    ipsum}\blindtext@endsentence   Lorem ipsum dolor sit amet,
    consectetuer adipiscing elit. Etiam lobortis facilisis sem. Nullam
    nec mi et neque pharetra sollicitudin. Praesent imperdiet mi nec
    ante. Donec ullamcorper, felis non sodales commodo, lectus velit
    ultrices augue, a dignissim nibh lectus placerat pede. Vivamus nunc
    nunc, molestie ut, ultricies vel, semper in, velit. Ut porttitor.
    Praesent in sapien. \blindmarkup{Lorem ipsum} dolor sit amet,
    consectetuer adipiscing elit. Duis fringilla tristique neque. Sed
    interdum libero ut metus. Pellentesque placerat. Nam rutrum augue a
    leo. Morbi sed elit sit amet ante lobortis sollicitudin. Praesent
    blandit blandit mauris. Praesent lectus tellus, \blindmarkup{aliquet
    aliquam}, luctus a, egestas a, turpis. Mauris lacinia lorem sit amet
    ipsum. Nunc quis urna dictum turpis accumsan semper.%
  }% \blindtext@text
}
%    \end{macrocode}
%
% Define counters for list environments.
%    \begin{macrocode}
\blind@addtext{french}{%
  \def\blindtext@count{%
    \ifcase\value{blind@listcount}\or
      Premier\or Deuxi\`eme\or Troisi\`eme\or Quatri\`eme\or Cinqui\`eme\or
      Sixi\`eme\or Septi\`eme\or Huiti\`eme\or Neuvi\`eme\or Dixi\`eme\or
      Onzi\`eme\or Douzi\`eme%
    \else
      L'autres%
    \fi
  }% \blindtext@count
  \def\blindtext@item{point dans une list}%
}%\addto\extrasfrench
%    \end{macrocode}
%
% Define title lines for French.
%    \begin{macrocode}
\blind@addtext{french}{%
  \def\blindtext@heading{Titres de niveau\xspace}%
  \def\blindtext@list{Lists}%
  \def\blindtext@listEx{Example pour une list\xspace}%
}%\addto\extrasfrench
%    \end{macrocode}
%
% Add the title for |\blindmathpaper|.
%    \begin{macrocode}
\blind@addtext{french}{%
    \def\blindtext@blindmath{Quelques textes avec des formules math\'ematiques.}%
}%\addto\extrasfrench
%    \end{macrocode}
%
%
% Define the bible-option text for french.
%    \begin{macrocode}
\ifblindbible
\blind@addtext{french}{%
  \def\blindtext@text{%
    L'\'Eternel Dieu dit au serpent: Puisque tu as fait cela, tu seras
    maudit entre tout le b\'etail et entre tous les animaux des champs,
    tu marcheras sur ton ventre, et tu mangeras de la poussi\`ere tous
    les jours de ta vie.
    Dieu dit \`a Mo\"ise: Je suis celui qui suis. Et il ajouta: C'est
    ainsi que tu r\'epondras aux enfants d'Isra\"el: Celui qui s'appelle
    'je suis'm'a envoy\'e vers vous.
    De la victime, il offrira en sacrifice consum\'e par le feu devant
    l'\'eternel: la graisse qui couvre les entrailles et toute celle qui
    y est attach\'ee,\ldots
    L'\'eternel parla \`a Mo\"ise, dans le d\'esert de Sina\"i, et dit:
    Ja\"ir, fils de Manass\'e, prit toute la contr\'ee d'Argob jusqu'\`a
    la fronti\`ere des Gueschuriens et des Maacathiens, et il donna son
    nom aux bourgs de Basan, appel\'es encore aujourd'hui bourgs de
    Ja\"ir.%
  }% \blindtext@text
  \def\blindtext@parstart{}%no change for bible option
}
\fi %\ifbible
%    \end{macrocode}
%
% Define the random-option text for french.
%    \begin{macrocode}
\ifblindrandom
  \PackageWarning{blindtext}{Option random not defined for french\MessageBreak}%
  \blind@addtext{french}{%
    \setcounter{blindtext}{1}
  }
\fi %option random
%    \end{macrocode}
%
% Define the pangram-option text for french.
%    \begin{macrocode}
\ifblindpangram
\blind@addtext{french}{%
    \setcounter{blindtext}{5}
    \def\blindtext@text{%
    \blind@countxx=1 %
    \loop  
      \ifcase\value{blind@pangramcount}%
Voyez le brick g\'eant que j'examine pr\`es du
wharf\blindtext@endsentence
\or Portez ce vieux whisky au juge blond qui fume\blindtext@endsentence
\or Buvez de ce whisky que le patron juge fameux
\or B\^achez la queue du wagon-taxi avec les pyjamas du
fakir\blindtext@endsentence
\or Voix ambigu\"e d'un c\oe ur qui au z\'ephyr pr\'ef\`ere les jattes
de kiwi\blindtext@endsentence
\or Monsieur Jack, vous dactylographiez bien mieux que votre ami
Wolf\blindtext@endsentence%
    \setcounter{blind@pangramcount}{-1}%
    \fi%
    \refstepcounter{blind@pangramcount}%
  \ifnum\blind@countxx<\value{blind@pangrammax}\advance\blind@countxx by 1 %
  \repeat%
  \setcounter{blind@pangrammax}{\value{blindtext}}%
  }% \blindtext@text  
  \def\blindtext@parstart{}%no change for pangram option
}
\fi %option pangram
%    \end{macrocode}
%
% ^^A %%%%%%%%%% End FrenchTexts %%%%%%%%%%%%%%%%
%

%
%<*packageend>
% \section{Thanks}
% Thanks to Heiko Oberdiek and Arno Trautmann for corrections (Version 1.8)
%
% Thanks to Andrea Bergschneider for her idea with math formulas inside the text
% (and Arno Trautmann for his cooperation in realizing it).
% Thanks to Dennis Heidsieck for his hint with polygloss.
% (Version 1.9)
%
% Thanks to Joan Queralt Gil for the Catalanian translation.
% Thanks to Felix Lehmann for corrections of the documentation and German and English blind texts.
% (Version 2.0)
%
% \Finale
% \PrintIndex
% \PrintChanges
% \end{document}
%</packageend>
